\documentclass{article}

\usepackage{tabularx}
\usepackage{booktabs}

\title{Problem Statement and Goals\\\progname}

\author{\authname}

\date{}

%% Comments

\usepackage{color}

\newif\ifcomments\commentstrue %displays comments
%\newif\ifcomments\commentsfalse %so that comments do not display

\ifcomments
\newcommand{\authornote}[3]{\textcolor{#1}{[#3 ---#2]}}
\newcommand{\todo}[1]{\textcolor{red}{[TODO: #1]}}
\else
\newcommand{\authornote}[3]{}
\newcommand{\todo}[1]{}
\fi

\newcommand{\wss}[1]{\authornote{blue}{SS}{#1}} 
\newcommand{\plt}[1]{\authornote{magenta}{TPLT}{#1}} %For explanation of the template
\newcommand{\an}[1]{\authornote{cyan}{Author}{#1}}

%% Common Parts

\newcommand{\progname}{OCRacle} % PUT YOUR PROGRAM NAME HERE
\newcommand{\authname}{Phillip Tran} % AUTHOR NAMES                  

\usepackage{hyperref}
    \hypersetup{colorlinks=true, linkcolor=blue, citecolor=blue, filecolor=blue,
                urlcolor=blue, unicode=false}
    \urlstyle{same}
                                


\begin{document}

\maketitle

\begin{table}[hp]
\caption{Revision History} \label{TblRevisionHistory}
\begin{tabularx}{\textwidth}{llX}
\toprule
\textbf{Date} & \textbf{Developer(s)} & \textbf{Change}\\
\midrule
2025-17-01 & Phillip Tran & Document created \\
\bottomrule
\end{tabularx}
\end{table}

\section{Problem Statement}

\subsection{Problem}

Researchers analyzing physical print documents such as newspapers, books, and
letters often need a means of digitizing the text in these documents. This
enables them to search and analyze the text data more efficiently. Especially
in the case of historical documents, digitizing the text can help preserve the
information contained in these documents.

Optical Character Recognition (OCR) is a technology that allows for the
extraction of text information from scanned documents, images, and other optical
formats where text may be present. This digitalization process enables
researchers to use computer programs to find trends and patterns in the
digitized text.


\subsection{Inputs and Outputs}

Input: A black and white image containing a single Latin alphabet character to
be recognized.

Output: The program's prediction of the Latin alphabet character in the image
and its confidence level.

\subsection{Stakeholders}

A researcher interested in digitizing text from an image would be the most
likely stakeholder for this tool.

\subsection{Environment}

The program will be compatible with Windows, MacOS, and Linux operating systems.
Any modern computers capable of running the operating systems mentioned above
should be able to run the program.

\section{Goals}

The program should be able to recognize Latin alphabet characters with an
overall accuracy of at least 80\%.

\section{Stretch Goals}

The program should be able to recognize Latin alphabet characters with an
overall accuracy of at least 90\%.

The program should be able to recognize Latin number characters with an overall
accuracy of at least 80\%.

\section{Challenge Level and Extras}

I expect this project to have a general challenge level. Although this project
has been done before, I expect that it will be challenging to achieve a high
level of accuracy in recognizing characters.

Since this is not considered a "research" project, I will be including a user
manual as an extra. This will help users understand how to use the program and
what to expect from it.

\newpage{}

\begin{enumerate}
    \item What went well while writing this deliverable? 
    \item What pain points did you experience during this deliverable, and how
    did you resolve them?
    \item How did you and your team adjust the scope of your goals to ensure
    they are suitable for a Capstone project (not overly ambitious but also of
    appropriate complexity for a senior design project)?
\end{enumerate}  

\end{document}