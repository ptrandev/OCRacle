\documentclass[12pt, titlepage]{article}

\usepackage[round]{natbib}
\usepackage{booktabs}
\usepackage{tabularx}
\usepackage{hyperref}
\hypersetup{
    colorlinks,
    citecolor=black,
    filecolor=black,
    linkcolor=red,
    urlcolor=blue
}
\usepackage[round]{natbib}

%% Comments

\usepackage{color}

\newif\ifcomments\commentstrue %displays comments
%\newif\ifcomments\commentsfalse %so that comments do not display

\ifcomments
\newcommand{\authornote}[3]{\textcolor{#1}{[#3 ---#2]}}
\newcommand{\todo}[1]{\textcolor{red}{[TODO: #1]}}
\else
\newcommand{\authornote}[3]{}
\newcommand{\todo}[1]{}
\fi

\newcommand{\wss}[1]{\authornote{blue}{SS}{#1}} 
\newcommand{\plt}[1]{\authornote{magenta}{TPLT}{#1}} %For explanation of the template
\newcommand{\an}[1]{\authornote{cyan}{Author}{#1}}

%% Common Parts

\newcommand{\progname}{OCRacle} % PUT YOUR PROGRAM NAME HERE
\newcommand{\authname}{Phillip Tran} % AUTHOR NAMES                  

\usepackage{hyperref}
    \hypersetup{colorlinks=true, linkcolor=blue, citecolor=blue, filecolor=blue,
                urlcolor=blue, unicode=false}
    \urlstyle{same}
                                


\begin{document}

\title{Verification and Validation Report: \progname} 
\author{\authname}
\date{\today}
	
\maketitle

\pagenumbering{roman}

\section{Revision History}

\begin{tabularx}{\textwidth}{p{3cm}p{2cm}X}
\toprule {\bf Date} & {\bf Version} & {\bf Notes}\\
\midrule
April 9, 2025 & 1.0 & Initial document creation\\
\bottomrule
\end{tabularx}

~\newpage

\section{Symbols, Abbreviations and Acronyms}

\renewcommand{\arraystretch}{1.2}
\begin{tabular}{l l} 
  \toprule		
  \textbf{symbol} & \textbf{description}\\
  \midrule 
  T & Test\\
  GHA & GitHub Actions\\
  \bottomrule
\end{tabular}\\

% \wss{symbols, abbreviations or acronyms -- you can reference the SRS tables if needed}

\newpage

\tableofcontents

\listoftables %if appropriate

\listoffigures %if appropriate

\newpage

\pagenumbering{arabic}

\section{Functional Requirements Evaluation}

All functional requirements were evaluated via automated tests using pytest.
All tests in the test suite passed successfully. To view the output of the
automated test suite, see Appendix \ref{sec:pytest-output}.

\section{Nonfunctional Requirements Evaluation}


\subsection{Performance}

T9 was supported by a test case in the automated test suite. The test case
ensured that the overall accuracy of the program exceeded 67.4\%, which is the
accuracy of the previous OAR project \citep{OARVnVReport}.

The \progname{} program achieved an accuracy of 90.55\% as indicated by the
Model Evaluation Output in Appendix \ref{sec:model-eval-output}, which is
a significant improvement over the previous OAR project.

\subsection{Usability}

T10 was not fully performed due to the time constraints of the project. However,
the project author performed a manual usability test of the program and found
that the documentation was accurate and complete.

\subsection{Maintainability}

To support T11, the code is run through the ruff linter, which checks for any
problems with the code formatting and automatically fixes any problems that it
can. In the final version of the code, there are no problems with the code.

To support T12, the Pyright static type checker was used to check for any
problems with the type annotations in the code. In the final version of the
code, there are no problems with the type annotations.

T13 was not fully performed due to the time constraints of the project. However,
the project author did perform a manual code review of the code based upon
the Source Code Checklist \citep{CodeChecklist} as specified by T13 and found
no major problems with the code in the final version.

\subsection{Portability}

To support T14, the automated Pytest test suite was run in a Windows, MacOS, and
Linux environment on GHA. All tests passed successfully in all environments,
validating the program's cross-platform compatibility.

% \section{Comparison to Existing Implementation}	

% This section will not be appropriate for every project.

\section{Unit Testing}

All unit tests were evaluated via automated tests using pytest. All tests in the
test suite passed successfully. To view the output of the automated test
suite, see Appendix \ref{sec:pytest-output}.

\section{Changes Due to Testing}

The following major changes were made after a peer demonstration:

\begin{itemize}
  \item The Accuracy Metrics Module was renamed the Model Evaluation Module in
  order to more clearly communicate the purpose of the module.
  \item The confusion matrix was renamed to the probability distribution in
  order to more clearly communicate the purpose of the model output.
\end{itemize}

% \wss{This section should highlight how feedback from the users and from 
% the supervisor (when one exists) shaped the final product.  In particular 
% the feedback from the Rev 0 demo to the supervisor (or to potential users) 
% should be highlighted.}

\section{Automated Testing}

The automated test suite was setup to run on GHA. The test suite is run
automatically on every commit to the main branch and for every pull request. The
automated tests are stored in the \href{https://github.com/ptrandev/OCRacle/tree/main/.github/workflows}{.github/workflows} directory.

\section{Trace to Requirements}

The following table outlines the traceability between the test cases and the
requirements outlined in the SRS document. Note: The table has been generated
using ChatGPT 4o. The output has been manually validated to ensure that it is
correct. \footnote{The following query was used: "Fill out the following traceability
matrix given the following table template and the information from this document: [table template and information from this document]."}

\begin{table}[h]
\centering
\begin{tabular}{|c|c|c|c|c|c|c|c|c|c|}
\hline
\textbf{Test Case} & \textbf{R1} & \textbf{R2} & \textbf{R3} & \textbf{R4} & \textbf{R5} & \textbf{NFR1} & \textbf{NFR2} & \textbf{NFR3} & \textbf{NFR4} \\ \hline
T1 & X &   &   &   &   &   &   &   &   \\ \hline
T2 & X &   &   &   &   &   &   &   &   \\ \hline
T3 & X &   &   &   &   &   &   &   &   \\ \hline
T4 &   & X &   &   &   &   &   &   &   \\ \hline
T5 &   &   & X &   &   &   &   &   &   \\ \hline
T6 &   &   &   & X &   &   &   &   &   \\ \hline
T7 &   &   &   & X &   &   &   &   &   \\ \hline
T8 &   &   &   &   & X &   &   &   &   \\ \hline
T9 &   &   &   &   &   & X &   &   &   \\ \hline
T10 &  &   &   &   &   &   & X &   &   \\ \hline
T11 &  &   &   &   &   &   &   & X &   \\ \hline
T12 &  &   &   &   &   &   &   & X &   \\ \hline
T13 &  &   &   &   &   &   &   &   & X \\ \hline
\end{tabular}
\caption{Test Cases to Requirements Matrix}
\label{tab:test-requirements-matrix}
\end{table}

\section{Trace to Modules}		

The following table outlines the traceability between test cases and modules
outlined in the MG \citep{MG}. Note: This table has been generated using ChatGPT
4o. The output has been manually validated to ensure that it is correct.
\footnote{The following query was used: "Here is an example of a table in LaTeX:
[table template]. Please create a new table for tracking the tracability of test cases and modules in the same table format. Here is the information you'll need to complete this: [list of modules and the tests that validate each one]."}

% \wss{Provide evidence that all of the modules have been considered.}

\begin{table}[h]
\centering
\begin{tabular}{|c|c|c|c|c|c|c|c|c|c|c|}
\hline
\textbf{Test Case} & \textbf{M1} & \textbf{M2} & \textbf{M3} & \textbf{M4} & \textbf{M5} & \textbf{M6} & \textbf{M7} & \textbf{M9} & \textbf{M10} \\ \hline
T1 &  &  & X &  &  &  &  &  &  \\ \hline
T2 &  &  & X &  &  &  &  &  &  \\ \hline
T3 &  &  & X &  &  &  &  &  &  \\ \hline
T4 &  &  &   &  & X &  &  &  &  \\ \hline
T5 &  &  & X &  &   &  & X &  &  \\ \hline
T6 &  &  &  &  &  & X &  &  &  \\ \hline
T7 &  &  &  &  &  & X &  &  &  \\ \hline
T8 &  &  &  &  &  &  & X &  &  \\ \hline
T9 &  &  &  &  &  &  &  & X &  \\ \hline
T10 & X & X & X & X & X & X & X & X & X \\ \hline
T11 & X & X & X & X & X & X & X & X & X \\ \hline
T12 & X & X & X & X & X & X & X & X & X \\ \hline
T13 & X & X & X & X & X & X & X & X & X \\ \hline
T14 & X & X & X & X & X & X & X & X & X \\ \hline
T15 &  &  &  & X &  &  &  &  &  \\ \hline
T16 &  &  &  &  &  &  &  &  & X \\ \hline
T17 &  &  &  &  &  &  &  &  & X \\ \hline
T18 &  &  & X &  &  &  &  &  &  \\ \hline
T19 &  &  & X &  &  &  &  &  &  \\ \hline
\end{tabular}
\caption{Traceability Matrix of Test Cases to Modules}
\label{tab:traceability-matrix}
\end{table}

\section{Code Coverage Metrics}

The following command was run to produce the code coverage metrics:
\footnote{Third-party libraries from the \texttt{src/libraries} directory were
omitted from the coverage report as this was considered out of scope for this
project. Additionally, \texttt{if \_\_name\_\_ == '\_\_main\_\_'} blocks were
also omitted since they are not executed when the module is imported.}

\begin{verbatim}
  coverage run -m pytest test/test.py && coverage report -m
\end{verbatim}

This produced the following output:

\begin{verbatim}
Name                   Stmts   Miss  Cover   Missing
----------------------------------------------------
src/data.py               14      0   100%
src/evaluation.py         17      0   100%
src/input.py              25      0   100%
src/model.py               5      0   100%
src/output.py              9      0   100%
src/preprocessing.py      11      0   100%
src/train.py              18      0   100%
test/test.py             112      0   100%
----------------------------------------------------
TOTAL                    211      0   100%
\end{verbatim}

\bibliographystyle{plainnat}
\bibliography{../../refs/References}

\section{Appendix}

\subsection{Pytest Test Suite Output}
\label{sec:pytest-output}

\begin{verbatim}
================== test session starts ==================
platform darwin -- Python 3.9.6, pytest-8.3.5, pluggy-1.5.0
-- /Users/phillip/Git/OCRacle/env/bin/python
cachedir: .pytest_cache
rootdir: /Users/phillip/Git/OCRacle
configfile: pyproject.toml
plugins: emoji-0.2.0, md-0.2.0, cov-6.1.1
collected 14 items                                                                                                                                                                                

test/test.py::testJpegAcceptance PASSED [  7%]
test/test.py::testPngAcceptance PASSED                                                                                                                                                      [ 14%]
test/test.py::testNonSupportedFormatRejection PASSED                                                                                                                                        [ 21%]
test/test.py::testImagePreProcessing PASSED                                                                                                                                                 [ 28%]
test/test.py::testCharacterPrediction PASSED                                                                                                                                                [ 35%]
test/test.py::testProbabilityVectorSum PASSED                                                                                                                                               [ 42%]
test/test.py::testProbabilityVectorLength PASSED                                                                                                                                            [ 50%]
test/test.py::testHumanReadableOutput PASSED                                                                                                                                                [ 57%]
test/test.py::testAccuracyMeasurement PASSED                                                                                                                                                [ 64%]
test/test.py::testModelTraining PASSED                                                                                                                                                      [ 71%]
test/test.py::testLoadTrainSubset PASSED                                                                                                                                                    [ 78%]
test/test.py::testLoadTestSubset PASSED                                                                                                                                                     [ 85%]
test/test.py::testFileNotFound PASSED                                                                                                                                                       [ 92%]
test/test.py::testInvalidDimensions PASSED                                                                                                                                                  [100%]

==================
warnings summary
==================
env/lib/python3.9/site-packages/urllib3/__init__.py:35
  /Users/phillip/Git/OCRacle/env/lib/python3.9/site-packages/urllib3/__init__.py:35: NotOpenSSLWarning: urllib3 v2 only supports OpenSSL 1.1.1+, currently the 'ssl' module is compiled with 'LibreSSL 2.8.3'. See: https://github.com/urllib3/urllib3/issues/3020
    warnings.warn(

test/test.py: 16 warnings
  /Users/phillip/Git/OCRacle/test/../src/libraries/emnist.py:226: DeprecationWarning: Conversion of an array with ndim > 0 to a scalar is deprecated, and will error in future. Ensure you extract a single element from your array before performing this operation. (Deprecated NumPy 1.25.)
    dim_size = int(numpy.frombuffer(data[offset : offset + 4], dtype=">u4"))

-- Docs: https://docs.pytest.org/en/stable/how-to/capture-warnings.html
================== 14 passed, 17 warnings in 59.12s ==================
\end{verbatim}

\subsection{Model Evaluation Output}
\label{sec:model-eval-output}

\begin{verbatim}
Loss: 0.3176746666431427
Accuracy: 0.9055288434028625
Confusion Matrix:
  tf.Tensor(
[[735   3   3   6  10   1   5   6   0   1   0   0   1   2   7   6   9   0
    0   0   2   0   0   1   0   2]
  [  5 754   0   4   3   0   4  13   1   1   0   6   0   1   3   0   1   1
    1   0   1   0   0   0   0   1]
  [  0   1 744   3  35   1   1   0   1   0   0   4   0   0   2   0   0   2
    2   1   2   0   1   0   0   0]
  [ 11   7   0 720   0   1   1   1   0   5   0   1   0   2  39   4   4   2
    0   0   0   0   0   0   0   2]
  [  1   1  13   0 769   0   2   0   2   0   0   3   0   0   2   2   1   2
    0   1   1   0   0   0   0   0]
  [  0   0   0   1   5 731   2   0   2   1   0   1   0   0   0  24   1   5
    4  23   0   0   0   0   0   0]
  [ 25  19   7   2  15   3 588   1   0   6   0   1   0   1   2   2 120   1
    4   1   0   0   0   0   2   0]
  [  8   1   0   1   0   1   0 739   1   1   6  13   4  10   0   0   0   1
    0   2   5   0   2   3   2   0]
  [  0   0   2   0   2   3   1   1 551  16   1 210   0   0   0   0   1   1
    1   0   0   1   1   2   0   6]
  [  1   1   0   8   0   1   4   0  29 734   0   3   0   0   0   0   1   0
    3  10   0   1   0   3   1   0]
  [  2   4   2   1   3   3   0  15   0   0 723   5   2   1   0   0   0   9
    0   6   4   2   0  16   1   1]
  [  0   1  11   0   0   1   0   4 158   1   0 621   0   0   1   0   1   0
    0   1   0   0   0   0   0   0]
  [  1   1   0   0   0   0   0   5   0   0   1   0 780   3   0   1   0   0
    0   1   3   0   2   2   0   0]
  [ 26   0   0   8   0   0   0  17   0   2   2   1  23 682   0   6   0  12
    0   2   5   2   8   4   0   0]
  [  3   0   1  12   2   0   0   0   0   0   0   0   0   1 775   2   2   1
    0   0   1   0   0   0   0   0]
  [  0   0   0   3   1   8   1   0   0   0   0   0   0   0   1 783   0   1
    0   1   0   0   0   1   0   0]
  [ 34   4   3   2   6   1  69   0   2   1   0   0   0   1   7   4 654   5
    0   1   2   0   1   0   3   0]
  [  9   4   2   0   5   4   0   0   2   0   4   2   1   1   0   3   1 727
    1   7   0  14   0   7   3   3]
  [  5   4   0   2   2   4  13   1   1  11   0   0   0   1   0   0   0   0
  756   0   0   0   0   0   0   0]
  [  1   1   1   0   3   2   0   0   3   1   1   2   0   0   1   1   1   2
    1 770   0   0   0   4   3   2]
  [  5   0   1   3   0   0   0   4   0   3   1   1   2   3   8   0   1   1
    1   0 737  25   2   1   1   0]
  [  0   0   0   2   0   0   0   0   0   1   0   1   0   1   0   0   0   8
    0   1  32 737   0   0  17   0]
  [  1   0   0   3   0   0   0   3   0   0   2   0   3   7   0   0   0   0
    0   1  12   1 765   2   0   0]
  [  4   0   0   2   0   1   1   1   0   1  10   1   0   0   0   1   1   2
    1   2   0   1   1 758  10   2]
  [  0   2   0   1   0   0   6   3   1   8   0   2   0   0   0   2   1   6
    1   4   2  14   0   8 739   0]
  [  4   1   1   2  10   1   5   1   3   0   0   0   0   0   0   1   1   1
    0   0   0   0   0   6   0 763]], shape=(26, 26), dtype=int32)
Class Accuracy:
  {'P': 0.97875, 'M': 0.975, 'O': 0.96875, 'T': 0.9625, 'E': 0.96125,
  'W': 0.95625, 'Z': 0.95375, 'X': 0.9475, 'S': 0.945, 'B': 0.9425, 'C': 0.93,
  'H': 0.92375, 'Y': 0.92375, 'U': 0.92125, 'V': 0.92125, 'A': 0.91875,
  'J': 0.9175, 'F': 0.91375, 'R': 0.90875, 'K': 0.90375, 'D': 0.9, 'N': 0.8525,
  'Q': 0.8175, 'L': 0.77625, 'G': 0.735, 'I': 0.68875}
\end{verbatim}

\end{document}